\documentclass[]{book}
\usepackage{lmodern}
\usepackage{amssymb,amsmath}
\usepackage{ifxetex,ifluatex}
\usepackage{fixltx2e} % provides \textsubscript
\ifnum 0\ifxetex 1\fi\ifluatex 1\fi=0 % if pdftex
  \usepackage[T1]{fontenc}
  \usepackage[utf8]{inputenc}
\else % if luatex or xelatex
  \ifxetex
    \usepackage{mathspec}
  \else
    \usepackage{fontspec}
  \fi
  \defaultfontfeatures{Ligatures=TeX,Scale=MatchLowercase}
\fi
% use upquote if available, for straight quotes in verbatim environments
\IfFileExists{upquote.sty}{\usepackage{upquote}}{}
% use microtype if available
\IfFileExists{microtype.sty}{%
\usepackage{microtype}
\UseMicrotypeSet[protrusion]{basicmath} % disable protrusion for tt fonts
}{}
\usepackage{hyperref}
\hypersetup{unicode=true,
            pdftitle={Network Analysis in R},
            pdfborder={0 0 0},
            breaklinks=true}
\urlstyle{same}  % don't use monospace font for urls
\usepackage{natbib}
\bibliographystyle{apalike}
\usepackage{color}
\usepackage{fancyvrb}
\newcommand{\VerbBar}{|}
\newcommand{\VERB}{\Verb[commandchars=\\\{\}]}
\DefineVerbatimEnvironment{Highlighting}{Verbatim}{commandchars=\\\{\}}
% Add ',fontsize=\small' for more characters per line
\usepackage{framed}
\definecolor{shadecolor}{RGB}{248,248,248}
\newenvironment{Shaded}{\begin{snugshade}}{\end{snugshade}}
\newcommand{\KeywordTok}[1]{\textcolor[rgb]{0.13,0.29,0.53}{\textbf{#1}}}
\newcommand{\DataTypeTok}[1]{\textcolor[rgb]{0.13,0.29,0.53}{#1}}
\newcommand{\DecValTok}[1]{\textcolor[rgb]{0.00,0.00,0.81}{#1}}
\newcommand{\BaseNTok}[1]{\textcolor[rgb]{0.00,0.00,0.81}{#1}}
\newcommand{\FloatTok}[1]{\textcolor[rgb]{0.00,0.00,0.81}{#1}}
\newcommand{\ConstantTok}[1]{\textcolor[rgb]{0.00,0.00,0.00}{#1}}
\newcommand{\CharTok}[1]{\textcolor[rgb]{0.31,0.60,0.02}{#1}}
\newcommand{\SpecialCharTok}[1]{\textcolor[rgb]{0.00,0.00,0.00}{#1}}
\newcommand{\StringTok}[1]{\textcolor[rgb]{0.31,0.60,0.02}{#1}}
\newcommand{\VerbatimStringTok}[1]{\textcolor[rgb]{0.31,0.60,0.02}{#1}}
\newcommand{\SpecialStringTok}[1]{\textcolor[rgb]{0.31,0.60,0.02}{#1}}
\newcommand{\ImportTok}[1]{#1}
\newcommand{\CommentTok}[1]{\textcolor[rgb]{0.56,0.35,0.01}{\textit{#1}}}
\newcommand{\DocumentationTok}[1]{\textcolor[rgb]{0.56,0.35,0.01}{\textbf{\textit{#1}}}}
\newcommand{\AnnotationTok}[1]{\textcolor[rgb]{0.56,0.35,0.01}{\textbf{\textit{#1}}}}
\newcommand{\CommentVarTok}[1]{\textcolor[rgb]{0.56,0.35,0.01}{\textbf{\textit{#1}}}}
\newcommand{\OtherTok}[1]{\textcolor[rgb]{0.56,0.35,0.01}{#1}}
\newcommand{\FunctionTok}[1]{\textcolor[rgb]{0.00,0.00,0.00}{#1}}
\newcommand{\VariableTok}[1]{\textcolor[rgb]{0.00,0.00,0.00}{#1}}
\newcommand{\ControlFlowTok}[1]{\textcolor[rgb]{0.13,0.29,0.53}{\textbf{#1}}}
\newcommand{\OperatorTok}[1]{\textcolor[rgb]{0.81,0.36,0.00}{\textbf{#1}}}
\newcommand{\BuiltInTok}[1]{#1}
\newcommand{\ExtensionTok}[1]{#1}
\newcommand{\PreprocessorTok}[1]{\textcolor[rgb]{0.56,0.35,0.01}{\textit{#1}}}
\newcommand{\AttributeTok}[1]{\textcolor[rgb]{0.77,0.63,0.00}{#1}}
\newcommand{\RegionMarkerTok}[1]{#1}
\newcommand{\InformationTok}[1]{\textcolor[rgb]{0.56,0.35,0.01}{\textbf{\textit{#1}}}}
\newcommand{\WarningTok}[1]{\textcolor[rgb]{0.56,0.35,0.01}{\textbf{\textit{#1}}}}
\newcommand{\AlertTok}[1]{\textcolor[rgb]{0.94,0.16,0.16}{#1}}
\newcommand{\ErrorTok}[1]{\textcolor[rgb]{0.64,0.00,0.00}{\textbf{#1}}}
\newcommand{\NormalTok}[1]{#1}
\usepackage{longtable,booktabs}
\usepackage{graphicx,grffile}
\makeatletter
\def\maxwidth{\ifdim\Gin@nat@width>\linewidth\linewidth\else\Gin@nat@width\fi}
\def\maxheight{\ifdim\Gin@nat@height>\textheight\textheight\else\Gin@nat@height\fi}
\makeatother
% Scale images if necessary, so that they will not overflow the page
% margins by default, and it is still possible to overwrite the defaults
% using explicit options in \includegraphics[width, height, ...]{}
\setkeys{Gin}{width=\maxwidth,height=\maxheight,keepaspectratio}
\IfFileExists{parskip.sty}{%
\usepackage{parskip}
}{% else
\setlength{\parindent}{0pt}
\setlength{\parskip}{6pt plus 2pt minus 1pt}
}
\setlength{\emergencystretch}{3em}  % prevent overfull lines
\providecommand{\tightlist}{%
  \setlength{\itemsep}{0pt}\setlength{\parskip}{0pt}}
\setcounter{secnumdepth}{5}
% Redefines (sub)paragraphs to behave more like sections
\ifx\paragraph\undefined\else
\let\oldparagraph\paragraph
\renewcommand{\paragraph}[1]{\oldparagraph{#1}\mbox{}}
\fi
\ifx\subparagraph\undefined\else
\let\oldsubparagraph\subparagraph
\renewcommand{\subparagraph}[1]{\oldsubparagraph{#1}\mbox{}}
\fi

%%% Use protect on footnotes to avoid problems with footnotes in titles
\let\rmarkdownfootnote\footnote%
\def\footnote{\protect\rmarkdownfootnote}

%%% Change title format to be more compact
\usepackage{titling}

% Create subtitle command for use in maketitle
\providecommand{\subtitle}[1]{
  \posttitle{
    \begin{center}\large#1\end{center}
    }
}

\setlength{\droptitle}{-2em}

  \title{Network Analysis in R}
    \pretitle{\vspace{\droptitle}\centering\huge}
  \posttitle{\par}
    \author{}
    \preauthor{}\postauthor{}
      \predate{\centering\large\emph}
  \postdate{\par}
    \date{2019-10-18}

\usepackage{booktabs}

\newenvironment{danger}
    {
    \hline\\
    }
    { 
    \\\\\hline
    }
    
\newenvironment{warning}
    {
    \hline\\
    }
    { 
    \\\\\hline
    }
    
\newenvironment{info}
    {
    \hline\\
    }
    { 
    \\\\\hline
    }
    
\newenvironment{try}
    {
    \hline\\
    }
    { 
    \\\\\hline
    }
\usepackage{booktabs}
\usepackage{longtable}
\usepackage{array}
\usepackage{multirow}
\usepackage{wrapfig}
\usepackage{float}
\usepackage{colortbl}
\usepackage{pdflscape}
\usepackage{tabu}
\usepackage{threeparttable}
\usepackage{threeparttablex}
\usepackage[normalem]{ulem}
\usepackage{makecell}
\usepackage{xcolor}

\begin{document}
\maketitle

{
\setcounter{tocdepth}{1}
\tableofcontents
}
\chapter*{Overview}\label{overview}
\addcontentsline{toc}{chapter}{Overview}

Materials for the 4 day Network Analysis course.

This course covers skills such as installing R, opening fies, working
with data using tidyverse, and making graphs. It also introduces network
analysis as a statistical concept.

\chapter{Starting with R}\label{starting-with-r}

{Welcome to the Course!}

\section{Overview}\label{overview-1}

Installing R and opening files

\textbf{In this session you will learn:}

\begin{enumerate}
\def\labelenumi{\arabic{enumi}.}
\tightlist
\item
  What is R?
\item
  How to install R
\item
  Where to get help
\end{enumerate}

\subsection{What is R?}\label{what-is-r}

For network analysis, you need two different bits of software, R and
RStudio. R is a programming language that you will write code in and R
Studio is an Integrated Development Environment (IDE) which makes
working with R easier.

\subsection{How To Install Base R}\label{how-to-install-base-r}

Install base R from \url{https://cran.rstudio.com/}. Choose the download
link for your operating system (Linux, Mac OS X, or Windows).

\subsection{How To Install R Studio}\label{how-to-install-r-studio}

Go to \url{https://rstudio.com} and download the RStudio Desktop (Open
Source License) version for your operating system under the list titled
\textbf{Installers for Supported Platforms.}

\subsection{Quiz}\label{quiz}

{Quickfire Questions}

We have put questions throughout to help you test your knowledge. When
you type in or choose the correct answer, the dashed box will change
color and become solid green.

\begin{itemize}
\tightlist
\item
  From the following options, how do you get R for this course?
  Installing Base R \& R Studio Installing R Studio Installing Base R
\end{itemize}

Explain This Answer!

R is the basic package. R Studio is an add-on that make R much easier to
use.

\subsection{Where to Get Help}\label{where-to-get-help}

\chapter{Working With Data in R}\label{working-with-data-in-r}

\section{Overview}\label{overview-2}

This is a basic introduction to R. The material is based on the data
skills course for MSc students at the University of Glasgow. Find lots
of useful resources here:
\url{https://gupsych.github.io/data_skills/01_intro.html} Please take a
look at these resources in your own time.

\subsection{Setting Working Directory}\label{setting-working-directory}

First things first, we will set the working directory. What this means
is that we need to tell R where the files we need are located. Think of
it just like when you have different projects, and you have seperate
folders for each project e.g.~research conducted in schools, research
conducted in the community and so on. When working on R, it's useful to
have all the data sets and files you need in one folder.

To set the working directory press session -\textgreater{} set working
directory -\textgreater{} choose directory and then select the folder
where the data sets we are working on are saved, and save this file in
the same folder as well. In other words- make sure your data sets and
scripts are all in the same folder.

\subsection{Code}\label{code}

RStudio generally has four panels: Current file, Console, Environment,
and Viewer. You can think of the console as a place to try things out,
and the file to write down ideas you want to stick around. Go to the
console and type

\begin{Shaded}
\begin{Highlighting}[]
\NormalTok{x <-}\StringTok{ }\DecValTok{1} \OperatorTok{+}\StringTok{ }\DecValTok{5}
\NormalTok{x}
\end{Highlighting}
\end{Shaded}

Notice how now the environment shows we have a Value x that is 6. We
have just created a variable. In the above, we would say ``the variable
x is assigned to 1 + 5'' or ``x gets 1 + 5''

\subsection{Functions \& Arguments}\label{functions-arguments}

We have already created some code. But what does it all mean?

Functions in R execute specific tasks and normally take a number of
arguments (if you're into linguistics you might want to think as these
as verbs that require a subject and an object). You can look up all the
arguments that a function takes by using the help documentation by using
the format \texttt{?function}. Some arguments are required, and some are
optional. Optional arguments will often use a default (normally
specified in the help documentation) if you do not enter any value.

As an example, let's look at the help documentation for the function
rnorm() which randomly generates a set of numbers with a normal
distribution. Just like the numbers in the graph below.

Open up R Studio and in the console, type the following code:

\begin{verbatim}
?rnorm
\end{verbatim}

The help documentation for \texttt{rnorm()} should appear in the bottom
right help panel. In the usage section, we see that \texttt{rnorm()}
takes the following form:

\begin{verbatim}
rnorm(n, mean = 0, sd = 1)
\end{verbatim}

In the arguments section, there are explanations for each of the
arguments. \texttt{n} is the number of observations we want to create,
\texttt{mean} is the mean of the data points we will create and
\texttt{sd} is the standard deviation of the set. In the details section
it notes that if no values are entered for mean and sd it will use a
default of 0 and 1 for these values. Because there is no default value
for n it must be specified otherwise the code won't run.

Now, try running the above code for 50 participants with a mean test
score of 3 and a standard deviation of 1.

Remember we are asking R to create \textbf{random} numbers here, so do
not worry if someone else has slightly different values.

I need a hint!

\texttt{n} in this case would be changed to 50.

\texttt{mean} should be changed to 3

\texttt{rnorm(n,\ mean\ =\ 3,\ sd\ =\ 1)}

now, try running the above code and see what happens.

\subsection{Tidyverse}\label{tidyverse}

However, we do not always want to use R to create random numbers, we
want to use it to analyse our own data which commonly resides in .csv or
.sav files. People have developed many different libraries to help us
work with such data. One of the most popular packages is tidyverse.

The Tidyverse is a collection of R packages with a common design ,
grammar, and data structure that makes analysis faster and easier.

The first time you want to use a package, you must first install the
package. \texttt{tidyverse} can be installed as follows.

\begin{verbatim}
install.packages("tidyverse")
\end{verbatim}

Once the package has been installed, any time you want to use the
package you use the following code. If you want to open a package other
than \texttt{tidyverse}, simply substitute the package name.

\begin{verbatim}
library("tidyverse")
\end{verbatim}

\chapter{Network Analysis}\label{network-analysis}

\section{Overview}\label{overview-3}

\section{Background}\label{background}

Recent thinking conceptulises mental wellbeing as comprising of
environmental, psychological and social factors. Psychologists wishing
to study all of these factors together may wish to consider a complexity
science perspective such as network analysis.

\section{What is a network?}\label{what-is-a-network}

A network is a set of nodes connected by a set of edges.

Several packages are used in the network analysis, including
\texttt{network}, \texttt{statnet}, \texttt{igraph} and \texttt{qgraph}.

qgraph was developed in the context of psychometrics approach by
Dr.~Sacha Epskamp and colleagues in 2012. We will be be working with
\texttt{qgraph}.

\chapter{Practical}\label{practical}

\section{Description of Data}\label{description-of-data}

We will work with a local dataset gathered from high school age
children.

\appendix


\chapter{References}\label{references}

\href{http://sachaepskamp.com/files/Cookbook.html}{Network Analysis
Cookbook - Also covers R introduction}

\href{https://psyteachr.github.io/}{We are grateful to PsyTeachR from
the University of Glasgow} for allowing us to build upon their open
source teaching materials.

\bibliography{book.bib,packages.bib}


\end{document}
