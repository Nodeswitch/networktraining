\documentclass[]{book}
\usepackage{lmodern}
\usepackage{amssymb,amsmath}
\usepackage{ifxetex,ifluatex}
\usepackage{fixltx2e} % provides \textsubscript
\ifnum 0\ifxetex 1\fi\ifluatex 1\fi=0 % if pdftex
  \usepackage[T1]{fontenc}
  \usepackage[utf8]{inputenc}
\else % if luatex or xelatex
  \ifxetex
    \usepackage{mathspec}
  \else
    \usepackage{fontspec}
  \fi
  \defaultfontfeatures{Ligatures=TeX,Scale=MatchLowercase}
\fi
% use upquote if available, for straight quotes in verbatim environments
\IfFileExists{upquote.sty}{\usepackage{upquote}}{}
% use microtype if available
\IfFileExists{microtype.sty}{%
\usepackage{microtype}
\UseMicrotypeSet[protrusion]{basicmath} % disable protrusion for tt fonts
}{}
\usepackage{hyperref}
\hypersetup{unicode=true,
            pdftitle={Network Analysis in R},
            pdfborder={0 0 0},
            breaklinks=true}
\urlstyle{same}  % don't use monospace font for urls
\usepackage{natbib}
\bibliographystyle{apalike}
\usepackage{longtable,booktabs}
\usepackage{graphicx,grffile}
\makeatletter
\def\maxwidth{\ifdim\Gin@nat@width>\linewidth\linewidth\else\Gin@nat@width\fi}
\def\maxheight{\ifdim\Gin@nat@height>\textheight\textheight\else\Gin@nat@height\fi}
\makeatother
% Scale images if necessary, so that they will not overflow the page
% margins by default, and it is still possible to overwrite the defaults
% using explicit options in \includegraphics[width, height, ...]{}
\setkeys{Gin}{width=\maxwidth,height=\maxheight,keepaspectratio}
\IfFileExists{parskip.sty}{%
\usepackage{parskip}
}{% else
\setlength{\parindent}{0pt}
\setlength{\parskip}{6pt plus 2pt minus 1pt}
}
\setlength{\emergencystretch}{3em}  % prevent overfull lines
\providecommand{\tightlist}{%
  \setlength{\itemsep}{0pt}\setlength{\parskip}{0pt}}
\setcounter{secnumdepth}{5}
% Redefines (sub)paragraphs to behave more like sections
\ifx\paragraph\undefined\else
\let\oldparagraph\paragraph
\renewcommand{\paragraph}[1]{\oldparagraph{#1}\mbox{}}
\fi
\ifx\subparagraph\undefined\else
\let\oldsubparagraph\subparagraph
\renewcommand{\subparagraph}[1]{\oldsubparagraph{#1}\mbox{}}
\fi

%%% Use protect on footnotes to avoid problems with footnotes in titles
\let\rmarkdownfootnote\footnote%
\def\footnote{\protect\rmarkdownfootnote}

%%% Change title format to be more compact
\usepackage{titling}

% Create subtitle command for use in maketitle
\providecommand{\subtitle}[1]{
  \posttitle{
    \begin{center}\large#1\end{center}
    }
}

\setlength{\droptitle}{-2em}

  \title{Network Analysis in R}
    \pretitle{\vspace{\droptitle}\centering\huge}
  \posttitle{\par}
    \author{}
    \preauthor{}\postauthor{}
      \predate{\centering\large\emph}
  \postdate{\par}
    \date{2019-10-16}

\usepackage{booktabs}

\newenvironment{danger}
    {
    \hline\\
    }
    { 
    \\\\\hline
    }
    
\newenvironment{warning}
    {
    \hline\\
    }
    { 
    \\\\\hline
    }
    
\newenvironment{info}
    {
    \hline\\
    }
    { 
    \\\\\hline
    }
    
\newenvironment{try}
    {
    \hline\\
    }
    { 
    \\\\\hline
    }
\usepackage{booktabs}
\usepackage{longtable}
\usepackage{array}
\usepackage{multirow}
\usepackage{wrapfig}
\usepackage{float}
\usepackage{colortbl}
\usepackage{pdflscape}
\usepackage{tabu}
\usepackage{threeparttable}
\usepackage{threeparttablex}
\usepackage[normalem]{ulem}
\usepackage{makecell}
\usepackage{xcolor}

\begin{document}
\maketitle

{
\setcounter{tocdepth}{1}
\tableofcontents
}
\chapter*{Overview}\label{overview}
\addcontentsline{toc}{chapter}{Overview}

Materials for the 2.5 day Network Analysis course.

This course covers skills such as installing R, opening fies, data
wrangling with tidyverse, and data visualisation with ggplot2. It also
introduces network analysis as a statistical concept.

\chapter{Starting with R}\label{starting-with-r}

{Welcome to the Course!}

\section{Overview}\label{overview-1}

Installing R and opening files

\textbf{In this session you will learn:}

\begin{enumerate}
\def\labelenumi{\arabic{enumi}.}
\tightlist
\item
  What is R?
\item
  How to install R
\item
  How to open files.
\item
  How to maniuplate data and save scripts.
\end{enumerate}

\subsection{What is R?}\label{what-is-r}

\subsection{Advantages of using R}\label{advantages-of-using-r}

{Quickfire Questions}

We have put questions throughout to help you test your knowledge. When
you type in or choose the correct answer, the dashed box will change
color and become solid green.

\begin{itemize}
\tightlist
\item
  From the following options, what is R? Statistical Programming
  Language Really Fun Cats
\end{itemize}

Explain This Answer!

People can click here to get answer explained.

\chapter{Working With Data in R}\label{working-with-data-in-r}

\section{Overview}\label{overview-2}

\chapter{Network Analysis}\label{network-analysis}

\section{Overview}\label{overview-3}

\section{Background}\label{background}

Recent thinking conceptulises mental wellbeing as comprising of
environmental, psychological and social factors. Psychologists wishing
to study all of these factors together may wish to consider a complexity
science perspective such as network analysis.

\section{What is a network?}\label{what-is-a-network}

A network is a set of nodes connected by a set of edges.

Several packages are used in the network analysis, including
\texttt{network}, \texttt{statnet}, \texttt{igraph} and \texttt{qgraph}.

qgraph was developed in the context of psychometrics approach by
Dr.~Sacha Epskamp and colleagues in 2012. We will be be working with
\texttt{qgraph}.

\chapter{Practical}\label{practical}

\section{Description of Data}\label{description-of-data}

We will work with a local dataset gathered from high school age
children.

\appendix


\chapter{References}\label{references}

\href{http://sachaepskamp.com/files/Cookbook.html}{Network Analysis
Cookbook - Also covers R introduction}

\href{https://psyteachr.github.io/}{We are grateful to PsyTeachR from
the University of Glasgow}

\bibliography{book.bib,packages.bib}


\end{document}
